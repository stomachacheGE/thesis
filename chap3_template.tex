\chapter{Drittes Kapitel}


Dies ist ein Text, der nur ein Beispiel darstellt. Dies ist ein Text, der nur ein 
Beispiel darstellt. Dies ist ein Text, der nur ein Beispiel darstellt.

Ein Zitat: \cite{fli:mult}.

Hier ein Verweis auf einen anderen Abschnitt: \ref{sec:abschnittlabel}.

Hier ein Verweis auf Abbildung \ref{fig:bildlabel_1}.

\index{Bild einbinden!ohne Makro}

\begin{figure}[ht]
	\centerline{\includegraphics[scale=0.5]{figures/dummy}}
	\caption{Eine Dummy Abbildung. Eine lange Bildunterschrift. Eine lange Bildunterschrift. Eine lange Bildunterschrift. Eine lange Bildunterschrift. Eine lange Bildunterschrift.}  
	\label{fig:bildlabel_1}
\end{figure}

\begin{figure}[ht]
	\centerline{\includegraphics[width=1.0\columnwidth]{figures/dummy}}
	\caption{Noch eine Dummy Abbildung.}    
	\label{fig:bildlabel_2}
\end{figure}

Dies ist ein Text, der nur ein Beispiel darstellt. Dies ist ein Text, der nur ein 
Beispiel darstellt. Dies ist ein Text, der nur ein Beispiel darstellt.
Dies ist ein Text, der nur ein Beispiel darstellt. Dies ist ein Text, der nur ein 
Beispiel darstellt. Dies ist ein Text, der nur ein Beispiel darstellt.


%\clearpage
%\pagebreak

\section{Abschnitt eins}

\label{sec:abschnittlabel}

Dies ist ein Text, der nur ein Beispiel darstellt. Dies ist ein Text, der nur ein 
Beispiel darstellt. Dies ist ein Text, der nur ein Beispiel darstellt.
Dies ist ein Text, der nur ein Beispiel darstellt. Dies ist ein Text, der nur ein 
Beispiel darstellt. Dies ist ein Text, der nur ein Beispiel darstellt.

\subsection{Unterabschnitt eins-zwei}

Dies ist ein Text, der nur ein Beispiel darstellt. Dies ist ein Text, der nur ein 
Beispiel darstellt. Dies ist ein Text, der nur ein Beispiel darstellt.
Dies ist ein Text, der nur ein Beispiel darstellt. Dies ist ein Text, der nur ein 
Beispiel darstellt. Dies ist ein Text, der nur ein Beispiel darstellt.


\section{Impulsformer - Wurzel-Cosinus-Rolloff Filter}

F�r die Nachrichten�bertragung werden Sende- und Empfangsfilter ben�tigt, die der 
Matched-Filter Bedingung gen�gen. Die Faltung des Sende- mit dem Empfangsfilter sollte 
ferner eine Cosinus-Rolloff-Charakteristik aufweisen. F�r die Sende- und Empfangsfilter 
werden daher Filter mit Wurzel-Cosinus-Rolloff-Charakteristik eingesetzt.

Die zeitkontinuierliche Version eines Cosinus-Rolloff Filters hat den folgenden 
Frequenzgang

\begin{equation}
H_{CR0}(j\omega) = 
\begin{cases}
1 & \mbox{f�r} \; \frac{\mid \omega \mid}{\omega_c} \leq 1-r \\
\frac{1}{2} \biggl(1 + \cos\biggl[ \frac{\pi}{2r}( \frac{\omega}{\omega_c} - 
(1-r))\biggr]  \biggl)
& \mbox{f�r} \; 1-r \leq \frac{\mid \omega \mid}{w_c} \leq 1+r \\
0 & \mbox{f�r} \; \frac{\mid \omega \mid}{\omega_c} \geq 1+r 
\end{cases}
\label{eq:H_CRO}
\end{equation}



\section{pdf features}

Link auf E-Mail:
\href{mailto:dafx2002@unibw-hamburg.de}{dafx2002@unibw-hamburg.de}

Link auf webpage: \\
DAFx main page: \href{http://www.dafx.de}{http://www.dafx.de}

Grafik mit Link auf webpage:
\begin{figure}[ht]
  \centerline{
	\href{http://www.dafx.de}{\includegraphics[width=1.0\columnwidth]{figures/dummy}}
	}
	\caption{Eine Dummy Abbildung mit Link.}    
	\label{fig:bildlabel_3}
\end{figure}



\clearpage

\section{Systemeigenschaften}

Im folgenden sind die verwendeten Parameter mit Zahlenwerten zusammengefasst:
\begin{itemize}
\item Symbol�bertragungsrate $f_T = 1/T = 2 f_N = 24 \; \mbox{kHz}$ \\
\item Bit�bertragungsrate bei verwendeter QPSK-Modulation $f_B = 2 f_T = 48 \; 
\mbox{kBit/s}$ \\
\item Symboldauer $T = \frac{1}{2 f_N} = 41.67 \; \mu$s \\
\item Nyquistfrequenz $f_N = 12 \; \mbox{kHz}$ \\
\item Rolloff-Faktor $r = 0.5$ \\
\item Parameter $M$ gibt den Faktor der �berabtastung an, $M = 2,3,4...$. \\ Bei 
verwendeter vierfacher �berabtastung $M = 4$. \\
\item Bandbreite im �bertragungsband (siehe Abbildung \ref{fig:bildlabel_3}) 
\begin{align}
B & = 2 \; (1+r) f_N \\
& = (1+r) f_T \\
& = 36 \; \mbox{kHz}
\end{align}
\item mindestens notwendige Abtastfrequenz nach Abw�rtsmischung ins Basisband 
\begin{align}
f_A & = 2 \; [2 \; f_N] \geq 2 \; [(1+r) \; f_N] \\
& = \frac{2}{T} = 2 f_T \\
& = 48 \; \mbox{kHz}
\end{align}
\item Abtastfrequenz bei verwendeter vierfacher �berabtastung nach Abw�rtsmischung ins 
Basisband 
\begin{align}
f_A & = 4 \; [2 \;f_N] \geq 4 \; [(1+r)\; f_N] \\
& = \frac{4}{T} = 4 f_T \\
& = 96 \; \mbox{kHz}
\end{align}
\item Tr�gerfrequenz $f_0 = 800 \; \mbox{MHz} \; / \; 2 \; \mbox{GHz}$ \\
\end{itemize}

