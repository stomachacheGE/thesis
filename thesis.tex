\NeedsTeXFormat{LaTeX2e}

% Rahmenumgebung f�r Studien Diplomarbeiten
% Erstellt von Oomke Weikert und Florian Keiler
% �nderungen im file _changelog.txt

% Titel und Autor der Arbeit unten im 
% \hypersetup Kommando �ndern
% ...und nat�rlich auf der Titelseite (title.tex bzw. title_en.tex)

% um pdf-file zu erzeugen:
% compilieren mit:
% pdflatex 
% bibtex
% pdflatex  
% pdflatex 
% (Hinweis: Das pdf-file darf bei Aufruf von pdflatex
% nicht im Adobe Reader ge�ffnet sein. Wenn man das
% pdf-file mit Ghostview �ffnet, muss es nicht geschlossen
% werden, und man kann dort die gerade bearbeitete Seite
% offen lassen)

% Im figures Ordner m�ssen die Bilder z.B. in pdf oder jpg Format liegen,
% mit pdflatex k�nnen *keine* eps Bilder benutzt werden.
% Das pdf file darf beim Kompilieren nicht 
% im Acrobat Reader ge�ffnet sein!!!

% um ps-file zu erzeugen:
% compilieren mit:
% latex 
% bibtex
% latex 
% latex 
% dvips

% dvips Aufruf f�r Type-1 Schriften: 
% dvips -t a4 -Ppdf 
% (alte GhostScript-Version: dvips -t a4 -Ppdf -G0) 
% Umwandeln in pdf mit Acrobat Distiller 
% oder mit ps2pdf (in GhostScript enthalten)


\documentclass[a4paper, twoside, 12pt, openright]{report}

\newif\ifmakeindex
\makeindextrue % generate index
%\makeindexfalse % don't generate index

\ifmakeindex
% fuer Stichwortverzeichnis
\usepackage{makeidx}

% Stichwortverzeichnis erstellen
\makeindex
\fi


\newif\ifenglish
\englishtrue   % english document
%\englishfalse  % german document

\usepackage[margin=1cm,format=hang,font=small,labelfont=bf,textfont=sl]{caption}
% angepasste Bildunterschriften
% Doku siehe beigef�gtes pdf-file

\usepackage{subfig}

\usepackage[english]{babel}

\usepackage[latin1]{inputenc}
% Unterstuetzen von deutschen Umlauten

\usepackage{t1enc}
% Verwenden von DC-Fonts (erlaubt richtiges Trennen
% auch in Worten mit Umlauten)

%\usepackage{times} 
  % Saves a lot of ouptut space in PDF...
  % ..after conversion with ps2pdf or Adobe Distiller	 
	% �ndert aber die Schriftart! (Times statt computer modern roman)
\usepackage{courier} % fett *und* typewriter nur damit m�glich?
% �ndert normale Schrift nicht!

\usepackage{cite}
% Automatisches Zusammenfassen von Literaturstellen

\usepackage{fancyheadings}	
%STY-file fancyheadings.sty nicht standardm��ig in miktex enthalten
%evtl. durch fancyhdr ersetzen

\usepackage{float}
%This package improves the interface for defining floating objects such
%as figures and tables in LaTeX.  It adds the notion of a `float style'
%that governs appearance of floats.  New kinds of floats may be defined
%using a \newfloat command analogous to \newtheorem.  This style option
%also incorporates the functionality of David Carlisle's style option
%`here', giving floating environments a [H] option which means `PUT IT
%HERE' (as opposed to the standard [h] option which means `You may put
%it here if you like').

\usepackage{verbatim}		
%This package reimplements the L A T E X verbatim and verbatim* envi- 
%ronments. In addition it provides a comment environment that skips any 
%commands or text between \begin{comment} and the next \end{comment}. 
%It also defines the command verbatiminput to input a whole file verbatim. 

\usepackage{amsmath}

%\usepackage{rotating}

\usepackage{array}
% wird benutzt von macro.tex f�r \Case

\usepackage{ifthen}
% wird benutzt von macro.tex \ifthenelse

\usepackage{longtable}
% f�r lange Tabellen gr��er als eine Seite mit \begin{longtable} .. \end{longtable}

\usepackage{varioref}
% Kommando \vref verweist mit Seitenzahl

\usepackage{readfile}
% ASCII-File einbinden mit
% \readit{fftdb.m}{\tt} 
% 2. Argument (\tt) gibt zu benutzende Schriftart an 
\usepackage{shapepar}
\usepackage{setspace}
\usepackage{colortbl}
\usepackage{color}
\usepackage{xcolor} 

\usepackage{listings}

\lstloadlanguages{}

\lstloadlanguages{C,C++,Java,Matlab,HTML,TeX,XML}

\definecolor{mygreen}{RGB}{28,172,0} % color values Red, Green, Blue
\definecolor{mylilas}{RGB}{170,55,241}
\definecolor{theme_color}{HTML}{C50042}


%\usepackage{subcaption}
%\lstset{}% restore default
%# title={Titel}:  gibt einen Titel zur Umgebung an (erscheint �ber dem Code zentriert)
%# caption={Titel}:  
%# label=Name:

\lstset{
				frame=lines, %Umrandung (single|none|shadowbox|lines|bottomline|topline|leftline)
				framerule=1pt, %Rahmenbreite
				tabsize=4, %Anzahl der Zeichen f�r ein TAB
				backgroundcolor=\color{lightgray}, %Hintergrundfarbe
				emph={}, %hebt die angegebenen W�rter hervor
				emphstyle=\underbar, %unterstreicht hervorgehobene W�rter
				columns=fixed, %Zeichenabstand (fixed | flexible | fullflexible)
				lineskip=0pt, %Zeilenabstand
				basicstyle=\ttfamily,%\ttfamily,
				% identifierstyle=\color{black},
    %     commentstyle=\color{darkgreen},
    %     stringstyle=\color{viola},
    %     keywordstyle=\color{darkblue},
				% ndkeywordstyle=\color{black},
				% showspaces=false,
				% showtabs=false,
				% numbers=none, %Zeilennummern (none|left|right)
				% %numbertype=\ttfamily,
				% breaklines=true,
    %     captionpos=b,
    %     extendedchars=false
    language=Matlab,%
    %basicstyle=\ttfamily\fontsize{8}{5}\selectfont,
    breaklines=true,%
    morekeywords={matlab2tikz},
    keywordstyle=\color{blue},%
    morekeywords=[2]{1}, keywordstyle=[2]{\color{black}},
    identifierstyle=\color{black},%
    stringstyle=\color{mylilas},
    commentstyle=\color{mygreen},%
    showstringspaces=false,%without this there will be a symbol in the places where there is a space
    %numbers=left,%
    %numberstyle={\tiny \color{black}},% size of the numbers
    %numbersep=9pt, % this defines how far the numbers are from the text
    %emph=[1]{for,end,break},emphstyle=[1]\color{red}, %some words to emphasise
    %emph=[2]{word1,word2}, emphstyle=[2]{style},  
}

%\usepackage{fancybox}
%\usepackage{theorem}
%\usepackage{amsbsy}
%\usepackage{nomencl}        % unterst�tzt Symbolverzeichnisse
%\usepackage{makeidx}
%\usepackage{multind}

\def\boxes{yes}
% \boxes == yes makes boxes around desired formulas by \Mbox
% wird benutzt von macros.tex
 

% pdf-tex settings:
% ------------------------
% detect automatically if run by latex or pdflatex

\usepackage{tikz}
\usetikzlibrary{arrows,calc,positioning,shapes}
\tikzset{
axis/.style={<->},
}
\usepackage{pgfplots}
\pgfplotsset{compat=1.5}
\newlength\figureheight 
\newlength\figurewidth 

\usepackage{bm}
\usepackage{amsfonts}
\usepackage{amsmath}
\usepackage{amssymb} %added by fu
\usepackage{textcomp} %added by fu

\usepackage{array} %added by fu
\usepackage{natbib}
\usepackage{setspace} %added by fu
\usepackage{tabu}
\usepackage{tabularx}
\usepackage{multirow} %added by fu
\usepackage{mathtools}
\usepackage{wrapfig}
\usepackage{longtable}
\usepackage{tabu}
\DeclarePairedDelimiter\abs{\lvert}{\rvert}%
\DeclarePairedDelimiter\norm{\lVert}{\rVert}
\usetikzlibrary{pgfplots.groupplots}
\usepgfplotslibrary{external} % make tikz compile only once
							  % added by fu
\tikzexternalize
%\usepgfplotslibrary{pggroupplots} % use group plots
\newcommand{\specialcell}[2][c]{%
  \begin{tabular}[#1]{@{}c@{}}#2\end{tabular}} %added by fu
  											   % for new lines in table cell

\newcommand{\cellpadding}[1]{%
  \vspace{-5pt} \\#1 \vspace{5pt}} %added by fu
  											   % for new lines in table cell

\usepackage{enumitem} %for better itemize
%\setlist{itemsep=8pt}

\usepackage{ifpdf}
%\newif\ifpdf  
%\ifx\pdfoutput\undefined
%   \pdffalse
%\else
%   \pdfoutput=1
%   \pdftrue
%\fi

\ifpdf % compiling with pdflatex
   %\usepackage[pdftex]{graphicx}    %%%%%%%%%%%%%%comment out by fu
   %\DeclareGraphicsExtensions{.pdf, .png, .jpg, .tikz} %%%%%%%%%%%%%%%%%comment out by fu
   \usepackage[pdftex,
	bookmarks,
	%colorlinks=false, % instead of colors, now boxes are used for links
	colorlinks=true,
	urlcolor=black, %blue,
	linkcolor=black, %red, %normal internal links
	citecolor=black, %green, %citation links
	%pagebackref, %link from references back to page of citation
	linktocpage, 
	% im Inhaltsverzeichnis Link auf Seitenzahl (sonst Probleme bei langen Zeilen)
	%breaklinks = true, % f�r Links l�nger als 1 Zeile
	%hypertexnames = false,  % f�r Links zu Figures?
	bookmarksopen, %open all bookmark folders
	bookmarksnumbered, %use section numbers with bookmarks
    pdfpagemode=UseOutlines, %show bookmarks
	% http://www.tex.ac.uk/cgi-bin/texfaq2html?label=pdfpagelabels
	plainpages=false, % eigene Seitenanker f�r r�mische/arabische Seitenzahlen
	pdfpagelabels, % im Abode Reader Seitenzahl als z.B. "iii (3 von 20)" anzeigen
%	pdfstartview=FitH
	pdfstartview=FitV
	]{hyperref}
    \pdfadjustspacing=1                %%% force LaTeX-like character spacing
    \pdfcompresslevel=9
	\pdfcatalog{                 
	% Catalog dictionary of PDF output. 
    % /PageMode /UseNone          
    % /URI (http://www.fi.muni.cz/)
	%       
	% pdfscreen-like setting might look like:
	%     /PageMode /none 
	%     /ViewerPreferences << 
	%         /HideToolbar true            
	%         /HideMenubar true 
	%         /HideWindowUI true 
	%         /FitWindow true 
	%         /CenterWindow true 
	%	
	% /PageMode determines how Acrobat displays the document on startup. 
	% The possibilities for the latter are explained below:
	% Supported /PageMode values.
	% /UseNone 		neither outline nor thumbnails visible
	% /UseOutlines 	outline visible
	% /UseThumbs 	thumbnails visible
	% /FullScreen 	full--screen mode
	% In full--screen mode, there is no menu bar, window controls, 
	% nor any other window present. The default setting is /UseNone.
	} %end of \pdfcatalog
\else % compiling with latex
  \usepackage[dvips]{graphicx}
%  \usepackage{color}
  \DeclareGraphicsExtensions{.eps}
  \usepackage[
    dvips,
%	ps2pdf,  % statt dvips, Unterschied???
	bookmarks,
  	colorlinks=false,  % for final paper without colors
%  	colorlinks=true,
	linktocpage, 	
	% im Inhaltsverzeichnis Link auf Seitenzahl (sonst Probleme bei langen Zeilen)
	%breaklinks = true, % f�r Links l�nger als 1 Zeile
	% funktioniert *nicht* korrekt f�r lange Zeilen mit gs < 7.05.3 ?
%	pdfstartview=FitH, % funktionfor different stakeholders of prostate cancer, e.g., patients, doctors and pharmaceutical companies.iert NICHT mit Adobe Distiller
	bookmarksopen, %open all bookmark folders
	bookmarksnumbered, %use section numbers with bookmarks
    pdfpagemode=UseOutlines, %show bookmarks
	pdfstartview=FitV
   ]{hyperref}
  % hyperrefs are active is the pdf file after conversion
  \hypersetup{
	pdfcreator  = {LaTeX with hyperref package},
	pdfproducer = {dvips + ps2pdf}
  }
\fi

% CHANGE TITLE AND AUTHOR !!!
\hypersetup{
  pdftitle={Studien/Diplomarbeit},
	pdfauthor={Autor},
	pdfsubject  = {Studien/Diplomarbeit, UniBwH, Professur ANT, \today},
	pdfkeywords = {}
}

% ------------------------

\input{macros.tex}

% make ',' an ordinary Symbol in decimal numbers
% Unterdr�ckung des Zwischenraums hinter einem Komma (z.B. in Dezimalzahlen)
% Steht im Mathemodus hinter dem Komma ein Leerzeichen, wird es als Trennzeichen
% (mit Zwischenraum) benutzt, sonst als Dezimalkomma.
 \mathchardef\CommaOrdinary="013B
 \mathchardef\CommaPunct   ="613B
 \mathcode`,="8000   % , im Math-Mode aktiv ("8000) machen
 {\catcode`\,=\active
  \gdef ,{\obeyspaces\futurelet\next\CommaCheck}}
 \def\CommaCheck{\if\space\next\CommaPunct\else\CommaOrdinary\fi}

\input{layout.tex}

\addto\captionsenglish{% 
	\renewcommand{\lstlistlistingname}{List of Sourcecodes} % default: "Listings"
	\renewcommand{\lstlistingname}{Sourcecode} % default: "Listing"
}



%\makeglossary

\begin{document}
\tikzstyle{block} = [rectangle, draw, text width=6em, text centered,  minimum height=3em]
\tikzstyle{block1} = [rectangle, draw, text width=9em, text centered,  minimum height=2em]
\tikzstyle{block2} = [rectangle, draw, text width=12em, text centered,  minimum height=2em]
\onehalfspacing
\include{colors} % several predefined colors - including the HSU corporate Design.

\pagenumbering{roman} % sonst gibt es 2 mal Seiten 1 und 2 (arabisch)
% so werden Probleme bei hyperlinks vermieden

%!TEX root = thesis.tex
%\def\voff{1cm}
%\addtolength{\voffset}{-\voff}

%\vspace{-2cm}
\pdfbookmark[1]{Title page}{sec:title}  % Bookmark im pdf file
\begin{titlepage}
\label{sec:title}

%\enlargethispage{\voff}
\large 
\begin{center}

\vspace{3cm}
% Unilogo mit Link auf Webseite
% \href{http://www.tuhh.de} 
% {
% \includegraphics[width = .45\textwidth]{figures/tuhh_logo.png} 
% \includegraphics[width = .45\textwidth]{figures/bosch_logo.png}
% }
\begin{figure}[ht]
\centering%%% not \center
%\subfigure[Figure A]{\label{fig:a}\includegraphics[width=60mm]{example-image-a}}
\subfloat{\includegraphics[width=.35\textwidth]{figures/tuhh_logo.png}}
\hspace{20pt}
\subfloat{\includegraphics[width=.45\textwidth]{figures/bosch_logo.png}}
\end{figure}

\vspace{2cm}

\begin{minipage}{.9\linewidth}
{\centering\Huge\bf Indoor Human Tracking based on Dynamic Models from Convolutional Neural Networks \par} 
% \par ist nötig für korrekten Zeilenabstand im Titel
\end{minipage}

\large 

%\vspace{1.5cm}
%{von\\[.3cm] {\bf \LARGE Autor}\\[1cm] 
%\huge	{ \bf \it Diplomarbeit} \\[1cm]}

\vspace{1.5cm}
{\huge{\bf \it Master Thesis} %\\[1cm]
}

\vspace{0.6cm}
{
\Large by \\[.5cm] 
{\bf \LARGE \huge Liangcheng Fu}\\[1.5cm]
}

\vfill
\begin{tabular}{ll}
Start date:& 01 August 2017\\
End date:& 30 January 2018\\
\\
First Examiner:&  Prof. Alexander Schl\"afer\\
Second Examiner:&  Prof. Dr.-Ing. habil. Udo Z\"olzer\\
Supervisor:&  Johannes D\"ollinger\\
\end{tabular}
\end{center}
\vspace{2cm}

\end{titlepage}

%\addtolength{\voffset}{\voff}

\thispagestyle{empty} % Titelr�ckseite (S. 2) ohne Seitennummer

\cleardoublepage

\newcounter{pageno}
\setcounter{pageno}{1} %titlepage = page 1, but pagenumber not printed
\addtocounter{pageno}{1}

% \cleardoublepage
% %!TEX root = project_report.tex
\pdfbookmark[1]{\abstractname}{sec:abstract}  % Bookmark im pdf file
\begin{abstract}
\label{sec:abstract}
\addtocounter{pageno}{1}
\setcounter{page}{\arabic{pageno}}
\thispagestyle{plain}
\begin{center}
\begin{minipage}[t]{\linewidth} 
\setlength{\parskip}{\parspacing}

% Inspired by cognitive science studies, Malisiewicz and his colleagues proposed ensemble of \textit{exemplar}-SVMs for object detection applications \cite{malisiewicz2011}. They reported that their algorithm  has comparable performance with other more complex methods. In this project work, further investigations on usage of \textit{exemplar}-SVMs are carried out. Precisely, \textit{exemplar}-SVMs are used for object classification in the context of maritime objects image set. Besides the Histogram of Gradients (HoG) feature used in their original work, Convolutional Neural Network (CNN) feature is also used in this project. The results show that, with linear SVM as benchmark, \textit{exemplar}-SVMs algorithm does not improve the performance in terms of classification accuracy. However, since the algorithm is able to retrieve images in the training set which have the same aspect ratio as the test image, it can be used for meta-data (e.g., segmentation and 3D model) transfer, which could be beneficial for overall scene understanding. 
\textbf{Background}: Thanks to the prevalence of Internet, nowadays many patients turn to online resources to seek information and emotional supports. Particularly, thanks to their easy access to discussion boards and chat rooms, Online Health Communities (OHCs) are more active. The posts that patients write on OHCs provide diverse and huge amount of information on patients' life. \\
\\
\textbf{Objective}: To explore how OHCs affect cancer patients' emotional state and what topics are been discussed. Moreover, investigate patients' opinions on keywords consisting of drugs, drug effects and therapies. \\
\\
\textbf{Method}: Web mining techniques are used to collect posts from OHCs and sentiment analysis with machine learning approach is applied on those posts. Besides, topic model is trained for automatically classifying posts to different topics. \\
\\
\textbf{Results}: Posts from two prostate cancer communities are collected. The first community is in English, consisting of 341,326 posts and the other one is in German with 69,089 posts. The best machine learning models achieve 81.66\% and 74.71\% accuracy for two communities respectively. The observations are : 1) For threads that start with negative sentiment, the vast majority of them (more than 92\% for both communities) have positive sentiment changes after their interactions with other community users. 2) Most of frustrating topics gain more sentiment change towards the positive side than normal and pleasant topics. 3) The average sentiment of threads containing each keyword is shown, which reflects patients' attitudes towards these drugs, drug effects and therapies. \\
\\
\textbf{Implications}:  1) Since sentiment of most negative originating threads change positively, this proves OHCs are effective to improve users' emotional state. 2) The topic model, together with machine learning model for extracting sentiment, can be used as a tool to monitor thread topic and its sentiment. The community administrator could make use of it to enhance community supports. 3) Patients' attitudes towards drugs, drug effects and therapies could be exploited by different stakeholders of prostate cancer, e.g., patients, doctors and pharmaceutical companies.   \\

\end{minipage}
\end{center}

\end{abstract}



% \cleardoublepage
% \begin{center}
\pdfbookmark[1]{Statement}{sec:statement}  % Bookmark im pdf file
{\Huge \bf Statement}
\label{sec:statement}
\end{center}
Hereby I do state that this work has been prepared by myself and with the help which is referred within
this thesis.

%Hiermit erkl�re ich, dass die vorliegende Arbeit von mir selbst�ndig und nur unter Verwendung der angegebenen Quellen und Hilfsmittel erstellt wurde.

\vspace{15cm}

Hamburg, 19. December 2016


% \cleardoublepage
% \include{foreword}

%\cleardoublepage
\input{fanc_h_2.tex} % set fancy headings
\include{toc}
\cleardoublepage

% Hier wird das Symbolzeichnis eingef�gt
% Falls es nicht gew�nscht ist, sind die folgenden sechs Zeilen auszukommentieren
% \cleardoublepage % to get correct page no. in TOC
% \phantomsection
% \markboth{\bfseries LIST OF SYMBOLS}{\bfseries LIST OF SYMBOLS}
% \addcontentsline{toc}{chapter}{\protect\numberline{List of Symbols}}
% \include{symbol_list}

\cleardoublepage
\setcounter{page}{1}
\renewcommand{\thepage}{\arabic{page}}

%============================================================================
%============================================================================
\onehalfspacing
% \cleardoublepage
%!TEX root = thesis.tex
\chapter{Introduction}


\cleardoublepage
%!TEX root = thesis.tex
\chapter{Literature Review} \label{chapter:2}

\section{Object Tracking}

\section{Neural Networks}

\cleardoublepage
%!TEX root = thesis.tex
\chapter{Background Knowledge}
%
\section{Bayesian Occupancy Filter (BOF)} 

\subsection{Bayesian Filtering}

\subsection{Bayesian Occupancy Filter Formulation}

\section{Fully Convolutional Neural Network}

\subsection{Densely Connected Convolutional Networks (DenseNets)}

\subsection{Fully Convolutional DenseNets}

\section{Metrics}

\cleardoublepage
%!TEX root = thesis.tex
\chapter{Implementation Details}

\section{Human Trajectory Simulation}

\section{Architecture of Neural Network}

\section{Implementation of BOFUM}

Before tracking starts, the occpancy and velocity probabilities are initialized uniformly, i.e.,

\[ P_c\{ O = occ \}=P_c\{O = nocc\}=0.5  P_c\{V= v \}=1/number \ of \ velocities \] 

\section{Hyperparameter Tuning}



\cleardoublepage
%!TEX root = thesis.tex
\chapter{Results}

\section{Overview of Datasets}

\subsection{Simulated Dataset}

\subsection{Real Dataset}

\section{Evaluation of Tracking Performance}

\subsection{Tracking on Simulated Data}

\subsection{Tracking on Real Data}
 

\cleardoublepage
%!TEX root = thesis.tex
\chapter{Outlooks}

\section{End to End Training}

\section{Future Work}
%Recurrent Tracking Network with Motion Model

\cleardoublepage
%\include{chap7}
%############# appendix #################
\cleardoublepage
\renewcommand{\chaptername}{\appendixname} %if name appears in header

% \begin{appendix}

% \chapter{Keyword Lists and Topic Names} \label{appendix2}
\setlength{\LTleft}{-20cm plus -1fill}
\setlength{\LTright}{\LTleft}
\begin{longtable}[c]{| p{.10\linewidth} | p{.60\linewidth} | p{.30\linewidth} |} 
    \hline
     & \textbf{Keywords} & \textbf{Topic name} \\
    \hline
    \hline
    1 &  bone lupron scan chemo months casodex hormone mets treatment started adt side scans effects trial radiation therapy oncologist drug shot & therapies and treatments \\
    \hline
    2 &  post thread edited gmt forum moderator healingwell link members information posted email posts posting site click journey community member threads & general forum management \\
    \hline
    3 &  doctor told back good time asked week call appointment today didn't husband results called he's doc office weeks thought wanted & appointments with doctors \\
    \hline
    4 &  pca treatment make men time medical patient life case people read years doctors information find good point patients things diseas & medical records  \\
    \hline
    5 &  surgery surgeon robotic gleason hospital open  good experience urologist diagnosed prostatectomy procedure post davinci age positive read vinci surgeons radiation  & surgery and surgery experience \\
    \hline
    6 &  months test month results year back years good post time surgery undetectable today radiation srt ago weeks level week blood & recall of experience with disease \\
    \hline
    7 &  pain back time day surgery days home night blood bladder hours catheter hospital morning side left week area weeks started & pain and blood-bladder issues \\
    \hline
    8 &  chat gfmph guys night home meet time trip join fun log room event meeting members plan awareness jeff word info & chat and trip \\
    \hline
    9 &  biopsy urologist years test gleason ago results dre age back doctor year months mri diagnosed found  scan blood negative normal & biopsy and MRI \\
    \hline
    10 &  rection viagra trimix surgery cialis sex pump injections erections months penis units injection good time dose sexual work side pills & erection and sex issue \\
    \hline
    11 &  room man car put water bag minutes home make virtue machine sling device made light hand wear house sitting guy & discussion on broad topics \\
    \hline
    12 &  url metformin radiation www research rad rel target kumc blank htm nucmed nofollow avodart marina failure doctor del rey lam & researches on radiation and medication \\
    \hline
    13 &  radiation treatment therapy side effects treatments imrt proton brachytherapy seeds hormone beam surger oncologist salvage brachy external hdr options months & side effects of radiation and other treatments \\
    \hline
    14 &  day surgery weeks catheter bladder incontinence days pads urine pad time night week dry months back post pee times remove & urine issues \\
    \hline
    15 &  www http usa article org news html link health interesting pca video watch htm net youtube site story articles thought  & news and articles from outside links \\
    \hline
    16 &  cells diet system immune exercise body cell supplements vitamin taking heart blood health eat weight healthy testosterone food found research & diet and exercise \\
    \hline
    17 &  insurance cost company drug pay medicare drugs approved health costs provenge cover year money plan fda order generic medical expensive  & insurance and payments \\
    \hline
    18 &  gleason lymph positive tumor report left pathology biopsy margins nodes invasion negative seminal tissue score margin cores grade surgery apex  & diagnosis \\
    \hline
    19 &  men patients study risk disease treatment clinical years survival screening trial therapy results percent group data studies early trials article & studies and clical trails \\
    \hline
    20 &  time good life years day year back feel hope great wife today people guys support love family friends things work & appreciations for life and families \\
    \hline
\caption{Topic keywords produced by \textit{MALLET} and the topic names.}
\end{longtable}

% %\cleardoublepage
% %!TEX root = thesis.tex
\chapter{Predefined Keyword List} \label{appendix3}
\vspace{-10pt}
\begin{table}[h]
\centering
\resizebox{.85\textwidth}{!}{
{\renewcommand{\arraystretch}{1}
   \begin{tabular}{| c | c | c | }
    \hline
     & \textbf {Keywords} & \textbf{Category} \\
    \hline
    \hline
    1 & \specialcell{Androgenentzugstherapie/Androgenentzug/ \\ Dreifache Hormonblockade/DHB} & \multirow{7}{*}{Therapy} \\ \cline{1-2}
    2 & \specialcell{Hormonentzug/Hormonablation/\\Antihormontherapie /AHT}  &  \\ \cline{1-2}
    3 & Androgendeprivationstherapie/ADT &  \\ \cline{1-2}
    4 & Robert Leibowitz MD/Robert Leibowitz & \\ \cline{1-2}
    5 & LHRH/LH-RH/GnRH/Gn-RH  & \\ \cline{1-2}
    6 & Antagonist  & \\ \cline{1-2}
    7 & Analoga & \\
    \hline
	8 & Bicalutamid/Casodex & \multirow{9}{*}{Drug} \\ \cline{1-2}
    9 & Abiraterone/Zytiga & \\ \cline{1-2}
    10 & Flutamid & \\ \cline{1-2}
    11 & Enzalutamid/Xtandi & \\ \cline{1-2}
    12 & Degarelix & \\ \cline{1-2}
    13 & Leuprorelin & \\ \cline{1-2}
    14 & Pamorelin & \\ \cline{1-2}
    15 & Buserelin & \\ \cline{1-2}
    16 & Lupron/Eligard & \\ \cline{1-2}
    \hline
    17 & Kastrationsniveau & \multirow{7}{*}{Effect} \\ \cline{1-2}
    18 & Kastration & \\ \cline{1-2}
	19 & Hitze & \\ \cline{1-2}
    20 & Hitzewallung & \\ \cline{1-2}
    21 & M\"digkeit/schlapp/m\"ude & \\ \cline{1-2}
    22 & Diabetes & \\ \cline{1-2}
    23 & Osteoporose & \\ \cline{1-2}
    \hline
    \end{tabular}
}
}
\caption{Predefined German keyword list for sentiment query.}
\end{table}

\newpage
%\vspace{50pt}
\begin{table}[p]
\centering
\resizebox{.85\textwidth}{!}{
{\renewcommand{\arraystretch}{1}
   \begin{tabular}{| c | c | c |}
    \hline
     & \textbf {Keywords} & \textbf{Category} \\
    \hline
    \hline
    1 & \specialcell{Total androgen blockage/TAB \\ /Triple Hormone blockade/blockage / \\ Triple androgen blocker }& \multirow{8}{*}{Therapy} \\ \cline{1-2}
    2 & \specialcell{Hormone therapy/treatment/blockage/HT}  &  \\ \cline{1-2}
    3 & Androgen deprivation therapy/ADT &  \\ \cline{1-2}
    4 & Robert Leibowitz MD/Robert Leibowitz & \\ \cline{1-2}
    5 & LHRH/LH-RH/GnRH/Gn-RH  & \\ \cline{1-2}
    6 & Antagonist  & \\ \cline{1-2}
    7 & Anti androgen/Anti-androgen  & \\ \cline{1-2}
    8 & Agonist & \\
    \hline
    9 & Lupron/Eligard & \\ \cline{1-2}
	10 & Bicalutamid/Casodex & \multirow{5}{*}{Drug} \\ \cline{1-2}
    11 & Abraxane & \\ \cline{1-2}
    12 & Cyclofosfamide & \\ \cline{1-2}
    13 & Enzalutamide/Xtandi & \\ \cline{1-2}
    14 & Degarelix & \\ \cline{1-2}
    \hline
    15 & Castrate level & \multirow{7}{*}{Effect} \\ \cline{1-2}
    16 & Castration & \\ \cline{1-2}
	17 & Heat & \\ \cline{1-2}
    18 & Hot flash & \\ \cline{1-2}
    19 & Fatigue & \\ \cline{1-2}
    20 & Diabetes & \\ \cline{1-2}
    21 & Osteoporosis & \\ \cline{1-2}
    \hline
    \end{tabular}
}
}
\caption{Predefined English keyword list for sentiment query.}
\end{table}


% \include{anhang3}
% \end{appendix}

% \cleardoublepage
% \phantomsection
% \addcontentsline{toc}{chapter}{\protect\numberline{List of Abbreviations}}
% %!TEX root = project_report.tex
\chapter*{List of Abbreviations}
\label{sec:abbreviations}
\markboth{\MakeUppercase{List of Abbreviations}}{\MakeUppercase{List of Abbreviations}} 
\begin{longtable}[t]{ll}
\arrayrulecolor{hsugrau}
\textbf{\textcolor{hsurot}{A}} &\\ \hline 
ANT & Allgemeine Nachrichtentechnik (Signal Processing and Communication)\\

  &\\ \textbf{\textcolor{hsurot}{C}} &\\  \hline 
 CNN & Convolutional Neural Network \\
 
   &\\ \textbf{\textcolor{hsurot}{E}} &\\  \hline 
 E-SVM & Exemplar-Support Vector Machine\\
 
    &\\ \textbf{\textcolor{hsurot}{G}} &\\  \hline 
 GUI & Graphical User Interface\\

  &\\ \textbf{\textcolor{hsurot}{H}} &\\  \hline 
HSU-HH & Helmut-Schmidt-University/University of the Federal Armed Forces Hamburg\\
HoG & Histogram of Gradients \\

  &\\ \textbf{\textcolor{hsurot}{I}} &\\  \hline 
ILSVRC & ImageNet Large Scale Visual Recognition Challenge\\

&\\ \textbf{\textcolor{hsurot}{M}} &\\  \hline 
MIT & Massachusetts Institute of Technology\\

&\\ \textbf{\textcolor{hsurot}{S}} &\\  \hline 
SVM & Support Vector Machine\\
SIFT & Scale-Invariant Feature Transform
\end{longtable}

%Softwareverzeichnis
% \cleardoublepage
% \phantomsection
% \addcontentsline{toc}{chapter}{\protect\numberline{List of Software}}
% \chapter*{List of Software}
\markboth{\MakeUppercase{List of Software}}{\MakeUppercase{List of Software}} 
\label{sec:software}
\begin{table}[htb]
	\centering
		\begin{tabular}{|l|l||l|l|}
		\hline
			\bfseries{Name} & \bfseries{Version} & \bfseries{URL} & \bfseries{Comment}\\
		\hline
		\hline
			MATLAB & R2015b & https://de.mathworks.com/ & MathsWorks\\
		\hline

		\hline
		\end{tabular}
	\caption{Used software.}
	\label{tab:VerwendeteSoftware}
\end{table}



% \cleardoublepage
% %########### Bibliography ##############
\clearpage % to get correct page no. in TOC
\phantomsection
\addcontentsline{toc}{chapter}{\protect\numberline{\bibname}}

\renewcommand{\chaptername}{} %if name appears in header

\renewcommand{\baselinestretch}{1}
\normalsize

% if enabling this command all references of the bib file are listed
% otherwise only the referenced books are listed in Bibliography section
%\nocite{*} 

%\bibliographystyle{unsrt}
%\bibliographystyle{alpha}
%\bibliographystyle{ALPHADIN}

%\bibliographystyle{IEEEtran} %IEEE Style
\bibliographystyle{plainnat}
	
% In literatur.bib sind die eigentlichen Literaturangaben enthalten
\bibliography{literatur} % requires file literatur.bib


\ifmakeindex
\cleardoublepage
\phantomsection
\addcontentsline{toc}{chapter}{\protect\numberline{Index}}
% Stichwortverzeichnis endgueltig anzeigen
\printindex
\fi

\end{document}
