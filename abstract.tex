%!TEX root = project_report.tex
\pdfbookmark[1]{\abstractname}{sec:abstract}  % Bookmark im pdf file
\begin{abstract}
\label{sec:abstract}
\addtocounter{pageno}{1}
\setcounter{page}{\arabic{pageno}}
\thispagestyle{plain}
\begin{center}
\begin{minipage}[t]{\linewidth} 
\setlength{\parskip}{\parspacing}

% Inspired by cognitive science studies, Malisiewicz and his colleagues proposed ensemble of \textit{exemplar}-SVMs for object detection applications \cite{malisiewicz2011}. They reported that their algorithm  has comparable performance with other more complex methods. In this project work, further investigations on usage of \textit{exemplar}-SVMs are carried out. Precisely, \textit{exemplar}-SVMs are used for object classification in the context of maritime objects image set. Besides the Histogram of Gradients (HoG) feature used in their original work, Convolutional Neural Network (CNN) feature is also used in this project. The results show that, with linear SVM as benchmark, \textit{exemplar}-SVMs algorithm does not improve the performance in terms of classification accuracy. However, since the algorithm is able to retrieve images in the training set which have the same aspect ratio as the test image, it can be used for meta-data (e.g., segmentation and 3D model) transfer, which could be beneficial for overall scene understanding. 
\textbf{Background}: Thanks to the prevalence of Internet, nowadays many patients turn to online resources to seek information and emotional supports. Particularly, thanks to their easy access to discussion boards and chat rooms, Online Health Communities (OHCs) are more active. The posts that patients write on OHCs provide diverse and huge amount of information on patients' life. \\
\\
\textbf{Objective}: To explore how OHCs affect cancer patients' emotional state and what topics are been discussed. Moreover, investigate patients' opinions on keywords consisting of drugs, drug effects and therapies. \\
\\
\textbf{Method}: Web mining techniques are used to collect posts from OHCs and sentiment analysis with machine learning approach is applied on those posts. Besides, topic model is trained for automatically classifying posts to different topics. \\
\\
\textbf{Results}: Posts from two prostate cancer communities are collected. The first community is in English, consisting of 341,326 posts and the other one is in German with 69,089 posts. The best machine learning models achieve 81.66\% and 74.71\% accuracy for two communities respectively. The observations are : 1) For threads that start with negative sentiment, the vast majority of them (more than 92\% for both communities) have positive sentiment changes after their interactions with other community users. 2) Most of frustrating topics gain more sentiment change towards the positive side than normal and pleasant topics. 3) The average sentiment of threads containing each keyword is shown, which reflects patients' attitudes towards these drugs, drug effects and therapies. \\
\\
\textbf{Implications}:  1) Since sentiment of most negative originating threads change positively, this proves OHCs are effective to improve users' emotional state. 2) The topic model, together with machine learning model for extracting sentiment, can be used as a tool to monitor thread topic and its sentiment. The community administrator could make use of it to enhance community supports. 3) Patients' attitudes towards drugs, drug effects and therapies could be exploited by different stakeholders of prostate cancer, e.g., patients, doctors and pharmaceutical companies.   \\

\end{minipage}
\end{center}

\end{abstract}

